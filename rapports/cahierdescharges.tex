\chapter{Cahier des charges}
\section{Premier prototype}
Il s'agit premièrement d'implémenter la fonction basique de notre programme: la résolution du cube. Nous allons donc lui fournir en entrée un fichier contenant les couleurs de toutes les facettes du cube. Le programme cherchera à le résoudre et renverra les mouvements nécessaires à sa résolution. Par exemple en utilisant la notation de Singmaster\cite{cite11}.

\section{L'affichage en 3D}
La prochaine étape du développement consiste à faire une interface homme machine. Il faut que l'utilisateur puisse visualiser les étapes successives de résolution du Rubik's Cube. Pour cela nous comptons afficher une fenêtre avec un bouton pour aller à l'étape suivante et une image en 3D de la configuration du cube que l'utilisateur devrait avoir dans les mains.

\section{\textit{Computer vision}}
Une fois que l'affichage a été réalisé, il faut pouvoir permettre de rentrer la configuration du cube au départ. Pour cela nous présenterons les différentes faces du cube à la caméra et un algorithme de traitement d'image\footnote{L'algorithme pourra être réalisé grâce à la bibliothèque OpenCV\cite{cite7}} sera capable de reconstituer le cube virtuellement.


