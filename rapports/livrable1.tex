\documentclass[a4paper]{report}

\usepackage[utf8]{inputenc}
\usepackage[T1]{fontenc}
\usepackage{hyperref}
\usepackage[francais]{babel}


\title{Rubik'INT \\ Livrable 1}
\author{Rémy \bsc{ZIRNHELD} \and Alban \bsc{MANZANO} \and Florian \bsc{GRANTE} \and Nicolas {BONNET}}
\date{7 février 2017}

\begin{document}

\maketitle

\tableofcontents

\chapter*{Introduction}
\addcontentsline{toc}{chapter}{Introduction}

Le projet de développement informatique nous a offert l'opportunité de créer le programme que nous voulions.
Divers sujets se présentaient à nous. Nous étions déterminés à trouver un projet nous permettant de mettre en œuvre plusieurs domaines de l'informatique.
Nous avons choisi le Rubik's Cube, un casse-tête fascinant: il s'agit d'un cube constitué de 3x3x3 sous-cubes colorés pouvant tourner selon trois axes.
Le Rubik's Cube se distingue par l'immense complexité de sa résolution contrastant avec sa simplicité apparente.
Chaque personne capable de résoudre le cube doit suivre une méthodologie stricte et faire preuve d'une certaine intuition.
L'objectif derrière le choix de ce célèbre casse-tête en tant que sujet est d'apprendre des domaines de l'informatique qui sont nouveau pour nous.

Le défi que nous nous sommes lancé est d'apprendre à une machine à résoudre ce casse-tête.
Pour cela nous devrons nous confronter à plusieurs problématiques.
La première, la plus évidente, est sa résolution. Nous allons avoir besoin de faire appel à des connaissances mathématiques et algorithmiques.
De plus nous voulons donner un aspect pratique à notre projet. Ainsi, l'utilisateur devra pouvoir réussir à résoudre son cube grâce au programme.
Il est donc indispensable d'élaborer une interface utilisateur.
Nous devrons aussi ajouter une solution simple pour reconnaitre la configuration du cube dans le programme. Nous utiliserons un algorithme de \textit{computer vision} sur un flux vidéo acquis en temps réel par une webcam.
Ainsi, ce projet regroupe les thématiques du traitement d'image, de l'algorithmie et du traitement d'image.

\chapter{Cahier des charges}
\section{Premier prototype}
Il s'agit premièrement d'implémenter la fonction basique de notre programme: la résolution du cube. Nous allons donc lui fournir en entrée un fichier contenant les couleurs de toutes les facettes du cube. Le programme cherchera à le résoudre et renverra les mouvements nécessaires à sa résolution. Par exemple en utilisant la notation de Singmaster\footnote{Plus d'information sur cette notation: \url{https://ruwix.com/the-rubiks-cube/notation/}}.
Différentes méthodes sont pour l'instant envisagées:
    -La première consiste à implémenter l'algorithme de résolution simple appris en général en premier par les humains. Elle consiste à envisager la résolution dans l'ordre suivant: d'abord une face, puis la première et seconde couronne et enfin la dernière face. \footnote {voir \url{https://www.francocube.com/cyril/rubik_index}} Pour résoudre le rubik's cube de cette manière, le programme devra être capable de reconnaitre la configuration dans laquelle se trouve le cube à un moment donné et lui
    appliquer une procédure, c'est-à-dire une suite prédéfinie de rotations. Les autres méthodes utilisées par les humains reposent sur ce même principe avec différentes configurations et procédures. Dans le cas des méthodes dites avancées, c'est-à-dire permettant de résoudre le Rubik's cube en un nombre plus petit de rotations, le nombre de configurations et de procédures à apprendre augmente drastiquement.
	-Un algorithme reposant sur une machine de boltzmann est également envisagée. Cet algorithme aurait une fonction d'évaluation de l'énergie du cube (plus le cube est proche de sa résolution, plus son énergie est faible) et chercherait à minimiser cette énergie via des rotations aléatoires ou des procédures aléatoires. Ces procédures auraient d'autant plus de chances d'être choisie que leur action réduit l'énergie du cube. Cet algorithme pourrait être utilisé conjointement avec le premier pour la résolution.
	-Un troisième algorithme dit en deux phases consiste à utiliser un algorithme A* (parcours d'un arbre en profondeur amélioré par un heuristique) pour placer les angles du cube au bon endroit lors de la première phase. La seconde phase consiste à utiliser réduire le nombre de rotation possible à un groupe de cardinal plus petit. Cet algorithme permet de résoudre le Rubik's Cube en une vingtaine de coups, mais est également plus dur à implémenter.

\section{L'affichage en 3D}
La prochaine étape du développement consiste à faire une interface homme machine. Il faut que l'utilisateur puisse visualiser les étapes successives de résolution du Rubik's Cube. Pour cela nous comptons afficher une fenêtre avec un bouton pour aller à l'étape suivante et une image en 3D de la configuration du cube que l'utilisateur devrait avoir dans les mains.

\section{\textit{Computer vision}}
Une fois que l'affichage a été réalisé, il faut pouvoir permettre de rentrer la configuration du cube au départ. Pour cela nous présenterons les différentes faces du cube à la caméra et un algorithme de traitement d'image sera capable de reconstituer le cube virtuellement.

\chapter{Interface}
Interface

\chapter{Structure du code}
Structure

\chapter*{Conclusion}
\addcontentsline{toc}{chapter}{Conclusion}

\end{document}
