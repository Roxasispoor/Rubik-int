\chapter{Cahier des charges}

\section{Besoins}
Notre application a pour but d'assister un utilisateur dans la résolution de son Rubik's Cube.
L'utilisateur rentrera la configuration du cube qu'il a à résoudre via une interface facile d'utilisation.
Le programme sera ensuite capable de lui indiquer la suite de manipulations qu'il doit effectuer pour résoudre le jeu.

L'interface se voudra facile d'utilisation, tant au niveau des menus qu'au niveau de l'entrée de la configuration du Rubik's.
L'utilisateur devra pouvoir suivre les étapes de résolution sans connaissance préalable.

\section{Les tâches}
\subsection{Le tableau}
\begin{tabularx}{\textwidth}{|X|X|X|}
  \hline
    \textbf{Fonctions} & \textbf{Tâches} & \textbf{Solution technique} \\
  \hline
  \multirow{5}*{Capturer l'état du RubiksCube} & Capturer un flux vidéo par webcam & WebcamPanel \\

  \hhline{~--}
  & Afficher la capture & WebcamPanel \\

  \hhline{~--}
  & Détecter les couleurs & Utilisation de l'espace HSV \\

  \hhline{~--}
  & Transformer un tableau de couleurs en object RubiksCube & Algorithme de correspondance \\
  \hline
  \multirow{2}*{Trouver les étapes de résolution} & Modéliser le RubiksCube & Utiliser un tableau de permutations \\

  \hhline{~--}
   & Résoudre le RubiksCube & Explorer un arbre avec des validateurs \\
  \hline
  \multirow{2}*{Faire une interface graphique} & Afficher des menus & Swing \\

  \hhline{~--}
   & Afficher le cube en 3D & OpenGL \\
  \hline
\end{tabularx}
\subsection{Le découpage}
Nous avons découpé notre projet en trois grands pôles: la partie graphique, la partie résolution et la partie \textit{computer vision}.
Chacun de ces trois pôles a eu des besoins spécifiques et a demandé l'utilisation de bibliothèques spécifiques.

