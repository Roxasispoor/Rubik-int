\chapter*{Conclusion}

Pour conclure, nous pouvons affirmer sans hésitation que ce projet nous a beaucoup appris.
Tout d'abord, nous avons du nous poser des questions de conception sur l'ensemble du projet et des différents éléments qui interagissent entre eux.
Pour se faire nous avons du apprendre à nous organiser en équipe, à faire un planning et une répartition équitable des tâches entre les membres. 
La conctrainte du code nous a poussé à uitliser utiliser des outils de versionning (tel que git) et à aquérir les bonnes pratiques associées.

Puis, à la phase d'implémentation, nous nous sommes confrontés à des problèmes qui nous semblaient simples en apparence mais qui se sont avérés compliqués algorithmiquement parlant.
Cela nous a permis d'approfondir nos connaissances sur le langage de programmation Java en utilisant des bibliothèques que nous n'avions jamais manipulé.
Nous avons été sensibilisés à la partie plus théorique de la programmation : nous avons du réfléchir sur le coût en temps de nombreuses fonctions.


Enfin, ce projet est venu enrichir notre expérience de travail en équipe.

Aujourd'hui nous sommes fiers de vous présenter notre programme, Rubik'INT. Nous espérons qu'il pourra servir et aider des personnes à apprendre à jouer au Rubik's cube.

